
\documentclass[12pt]{article}

%%%%%%%%%%%%%%%%%%%%%%%%%%%% PREAMBLE

\title{CS2510 Project Proposal:
An Analysis of Distributed Crowds}

\author{
		Lindsey Bieda \\
        Yann Le Gall \\
		Joe Cavanaugh \\
		University of Pittsburgh \\ 
        Department of Computer Science \\
}
\date{\today}
%%%%%%%%%%%%%%%%%%%%%%%%%%%%%

%---------------------------- packages
\usepackage[pdftex]{graphicx}
%\usepackage{graphicx}
%---------------------------- packages

\begin{document}

\maketitle


\section{Introduction}

We propose to create a simulation of \emph{Crowds}, which is a distributed
system that provides anonymity on the web. This system was proposed by
Michael K. Reiter and Aviel D. Rubin of AT\&T Research Labs in 1998 \cite{reiter}.

\subsection{Overview of Crowds}
In crowds, clients form a group, called a crowd. Crowd members are called
 \emph{jondos}, which is a play on \emph{John Doe}. When a jondo wishes to
make a web request to a particular server, the jondo either contacts
the server directly, or forwards the request to another jondo with
some probability. Likewise, when a crowd member receives a request from
another jondo, it either sends the request to the server or forwards it
to another jondo. As a result, random paths are
established. From the perspective of the server, the request could
have originated from any one of the crowd members. The response from the
server then travels in the reverse direction along the path.

A central node called the \emph{blender} manages the several aspects of
the system. The blender admits a group of new jondos to the crowd in a
batch, establishes random paths, and distributes other necessary
meta-data.

\subsection{Weaknesses}
The authors point out several potential issues with Crowds:

\begin{itemize}
	\item There is a tradeoff between performance and degree of privacy
	against collaborating jondos. Anonymity depends on the path
	path length, which depends on the probability of forwarding
	requests, but an increase in path length will increase the latency
	of web requests.
	\item Crowds provide implicit load balancing: as more clients
	join the crowd, the increased number of requests is spread out
	across more client machines. However, there is still only a single
	blender that is liable to become a bottleneck.
	\item SSL and client scripting pose obstacles to having the last node
	pre-fetch any embedded web requests.
	\item Firewalls may prevent communication between certain clients
	and servers.
	\item Crowds do not address denial of service, global eavesdroppers,
	and sender-receiver linkability.
\end{itemize}

\section{Project Goal}

The goal of this project is to analyze the tradeoffs between response time
and privacy for jondos. The identity of a jondo has the potential to be
discovered by malicious jodnos; several malicious jondos might collaborate
with each other in order to determine the originator of a message.

However this threat can be mitigated by increasing the number of legitimate
jondos in the crowd and by increasing the average length of forwarding
paths (which is controlled by the probability of forwarding, $p_f$).
Unfortunately, these measures increase the average response time of web
requests. With longer paths, the web requests must travel further to
reach their destination. Furthermore, a larger number of jondos increases the load on
the blender, especially at times when paths are established and routing
information is updated.

\section{Design and Implementation}

\subsection{Assumptions}

We will make several assumptions to accommodate the limited
time constraints and scope of this project.

\begin{itemize}
	\item The Crowds protocol calls for symmetric key encryption between
	jondos, but our simulation will not implement encrpytion and decryption.
	\item We will assume a reliable network. Specifically, we will not
	focus on lost or altered messages.
	\item Jondos do not crash. That is, the departure of a jondo is always
	announced.
	\item Finally, we will assume that the blender is always available.
	In other words, we will not consider the possibility that the blender
	crashes.
\end{itemize}

\section{Evaluation}

In this section we propose several metrics by which we will evaluate
and compare the tradeoffs between anonymity and response time.

To measure the anonymity of a jondo, Reiter and Rubin define 
a term called \emph{probable innocence}. From the attacker's point of
view, a jondo has probable innocence if the probability that she is the
originator of a request is less than or equal to the probability that
she is not the originator of a request. In short, a jondo is probably
innocent if there is smaller than $\frac{1}{2}$ chance that it is the
originator of some request.

Reiter and Rubin show that the originating jondo has probable innocence
against $c$ collaborators if
\[
	n \ge \frac{p_f}{p_f - \frac{1}{2}}(c+1)
\]

where $n$ is the total number of nodes in the crowd, and $p_f$ is
the probability of forwarding a request. Note that protection against
malicious jondos depends on the size of the crowd and the length of
the path. Thus, in our simulation we will analyze how changes in these
parameters influence the average response time of requests from jondos.


\begin{thebibliography}{9}

	\bibitem{reiter}
	  Reiter, Michael and Rubin, Aviel.
	  \emph{Crowds: Anonymity for Web Transactions}.
	  ACM Transactions on Information and System Security.
	  Vol. 1, No. 1, 1998.

\end{thebibliography}


\end{document}

